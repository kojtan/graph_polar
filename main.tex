
\documentclass[letterpaper,12pt]{article}
\usepackage[width=17cm]{geometry}
\usepackage[spanish]{babel}
\selectlanguage{spanish}
\usepackage[utf8]{inputenc}
\usepackage{mathrsfs}
\usepackage{enumerate}
\usepackage{mathtools}
\usepackage{pgf,tikz}
\usetikzlibrary{arrows}

\usepackage{multicol}
\setlength{\columnsep}{1cm}

\usepackage{mathptmx}
\usepackage[T1]{fontenc}


%\usepackage{mathpazo}
 \usepackage{amsmath,amsthm,amssymb,amsfonts, enumitem, fancyhdr, color, comment, graphicx, environ}
\pagestyle{fancy}
\setlength{\headheight}{65pt}

%\theoremstyle{definition}
\newtheorem{prob}{Problema}
\newtheorem{defi}{Definición}

%%%%%%%%%%%%%%%%%%%%%%%%%%%%%%%%%%%%%%%%%%%%%
%Fill in the appropriate information below
\lhead{\textbf{Matemática III}\\Hoja de problemas 5\\Ciclo Impar 2020 - 18/Marzo}  %replace with your name
\rhead{UES - F.CC.NN. y Mat.\\Escuela de Matemática\\por JuanJo Ramírez} %replace XYZ with the homework course number, semester (e.g. ``Spring 2019"), and assignment number.
%%%%%%%%%%%%%%%%%%%%%%%%%%%%%%%%%%%%%%%%%%%%%


%%%%%%%%%%%%%%%%%%%%%%%%%%%%%%%%%%%%%%
%Do not alter this block.
\begin{document}

\begin{center}
\Large \textbf{$\S$ Coordenadas polares}
\end{center}
\textbf{Indicación:} Resolver los siguientes problemas dejando claros los argumentos realizados para su resolución.
\section{Gráficas en coordenadas polares}
\begin{prob}
En los siguientes problemas, discutir la simetría y el lugar geométrico de cada ecuación.
\begin{multicols}{3}
\begin{enumerate}[label=\alph*)]
    \item $r=4\cos{\theta}$
    \item $r=-3\sin{\theta}$
    \item $\displaystyle r=2\cos\left(\theta+\frac{\pi}{6}\right)$
    \item $r=3(1-\cos{\theta})$
    \item $r=2(1+\sin{\theta})$
    \item $r=2-4\sin{\theta}$
    \item $r\sin{\theta}=1$
    \item $r^2=2\cos{\theta}$
    \item $r^2=2\sin{2\theta}$
    \item $r^2=\cos{2\theta}$
    \item $r=5\sin{4\theta}$
    \item $r=5\sin{2\theta}$
    \item $\displaystyle r=4\sin^2\left(\frac{\theta}{2}\right)$
    \item $r=\tan{\theta}$
    \item $r(2-\cos{\theta})=4$
    \item $r=2+\theta$
    \item $r=3-2\theta$
    \item $r=2e^\theta$
    \item $r=3e^{-2\theta}$
\end{enumerate}
\end{multicols}
\end{prob}
\section{Ecuaciones en \textit{coordendas cartesianas vrs coordenadas polares}}
\begin{prob}
En los siguientes problemas, las ecuaciones están dadas en coordenadas cartesianas. En cada caso, encontrar una ecuación en coordenadas polares que describa el mismo lugar geométrico.
\begin{multicols}{3}
\begin{enumerate}[label=\alph*)]
    \item $x=-2$
    \item $y=3$
    \item $x-y=0$
    \item $2x+3y=0$
    \item $2x+y\sqrt{2}=4$
    \item $xy=4$
    \item $x^2+y^2+2x-4y=0$
    \item $x^2+y^2+2x+6y=0$
    \item $y^2=2x$
    \item $x^2+y^2-2x=0$
    \item $x^2+y^2+4y+4=0$
\end{enumerate}
\end{multicols}
\end{prob}
\newpage
\begin{prob}
Encontrar en cada caso una ecuación polinómica en coordenadas cartesianas cuyo lugar geométrico contenga los puntos de la ecuación dada en coordenadas polares. Discutir las posibles soluciones extrañas.
\begin{multicols}{3}
\begin{enumerate}[label=\alph*)]
    \item $r=7$
    \item $\displaystyle \theta=\frac{\pi}{3}$
    \item $r=3\cos{\theta}$
    \item $r\cos{\theta}=5$
    \item $r\sin{\theta}=-2$
    \item $\displaystyle r\cos\left(\theta+\frac{\pi}{6}\right)=1$
    \item $r^2\sin{2\theta}=2$
    \item $r=2\sec{\theta}\tan{\theta}$
    \item $r(2-\cos{\theta})=2$
    \item $r=a\cos{3\theta}$. con $a\in \mathbb{R}$.
    \item $r=1-\cos{\theta}$
    \item $r=1-2\sin{\theta}$
    \item $r=a\cos{2\theta}$, con $a\in \mathbb{R}$.
\end{enumerate}
\end{multicols}
\end{prob}

\begin{prob}
La ecuación $r^2=\cos{4\theta}$ corresponde a una ecuación polinómica en $x$ y $y$. ¿Cuál es el grado de este polinomio?
\end{prob}

\begin{prob}
La ecuación $r^2=\cos(2n\theta)$, donde $n\in \mathbb{N}$, corresponde a un polinomio en $x$ y $y$. ¿Cuál es el grado de este este polinomio?
\end{prob}
\end{document}